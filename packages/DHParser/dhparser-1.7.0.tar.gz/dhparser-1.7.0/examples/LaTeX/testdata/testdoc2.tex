\documentclass[12pt, a4paper]{article}

\usepackage{microtype}
\usepackage[USenglish]{babel}
% \usepackage[utf8x]{inputenc}
% \usepackage{ucs}
\usepackage[backend=biber, bibencoding=utf8, citestyle=authoryear,
  sortlocale=en_US, natbib=true, url=false, doi=true, eprint=false]
  {biblatex}
\usepackage{csquotes}
\addbibresource{bibliography.bib}
\usepackage{eurosym}
\usepackage{setspace}
\usepackage{array}

%\usepackage{mathptmx}       % selects Times Roman as basic font
%\usepackage{helvet}         % selects Helvetica as sans-serif font
%\usepackage{courier}        % selects Courier as typewriter font
%\usepackage{type1cm}        % activate if the above 3 fonts are
                            % not available on your system
%\usepackage{makeidx}
%\usepackage{setspace}

\usepackage{ifpdf}
\ifpdf
\usepackage{xmpincl}
\usepackage[pdftex]{hyperref}
\hypersetup{
    colorlinks,
    citecolor=black,
    filecolor=black,
    linkcolor=black,
    urlcolor=black,
    unicode=true,
    bookmarksopen=true,     % Gliederung öffnen im AR
    bookmarksnumbered=true, % Kapitel-Nummerierung im Inhaltsverzeichniss anzeigen
    bookmarksopenlevel=1,   % Tiefe der geöffneten Gliederung für den AR
    pdfstartview=FitV,       % Fit, FitH=breite, FitV=hoehe, FitBH
    pdfpagemode=UseOutlines % FullScreen, UseNone, UseOutlines, UseThumbs
}
%\includexmp{Arnold_2017_Validation_from_a_Kuhnian_Perspective}
\pdfinfo{
  /Author (Eckhart Arnold)
  /Title (Simulation-Validation from a Kuhnian Perspective)
  /Subject (Discussion of the thesis that the validation of computer simulations constitutes a new pardigm of scientific validation)
  /Keywords (Computer Simulations, Validation of Simulations)
}
\fi

\begin{document}

\title{Validation of Computer Simulations from a Kuhnian Perspective}

\author{Eckhart Arnold, Bavarian Academy of Sciences and Humanities}

\date{August 2018}

\maketitle

\sloppy

\begin{center}
{\em erscheint in: Beisbart, C. \& Saam, N. J. (eds.), Computer
Simulation Validation  - Fundamental Concepts, Methodological
Frameworks, and Philosophical Perspectives, Cham: Springer 2019.}
\end{center}

\begin{abstract}
\singlespacing

While Thomas Kuhn's theory of scientific revolutions does not
specifically deal with validation, the validation of simulations can
be related in various ways to Kuhn's theory: 1) Computer simulations
are sometimes depicted as located between experiments and theoretical
reasoning, thus potentially blurring the line between theory and
empirical research. Does this require a new kind of research logic
that is different from the classical paradigm which clearly
distinguishes between theory and empirical observation? I argue that
this is not the case. 2) Another typical feature of computer
simulations is their being ``motley'' \citep{winsberg:2003} with
respect to the various premises that enter into simulations. A
possible consequence is that in case of failure it can become
difficult to tell which of the premises is to blame. Could this issue
be understood as fostering Kuhn's mild relativism with respect to
theory choice? I argue that there is no need to worry about relativism
with respect to computer simulations, in particular. 3) The field of
social simulations, in particular, still lacks a common understanding
concerning the requirements of empirical validation of simulations.
Does this mean that social simulations are still in a pre-scientific
state in the sense of Kuhn? My conclusion is that despite ongoing
efforts to promote quality standards in this field, lack of proper
validation is still a problem of many published simulation studies and
that, at least large parts of social simulations must be considered as
pre-scientific.


\begin{flushleft}
  {\bf Keywords}: Computer Simulations, Validation of Simulations,
  Scientific Paradigms
\end{flushleft}

\end{abstract}

\newpage

\tableofcontents

\onehalfspacing

\section{Introducxxxtion}

Thomas \citet{kuhn:1976} famously introduced the term {\em paradigm}
to characterize the set of background beliefs and attitudes shared by
all scientists of a particular discipline. According to Kuhn these
beliefs and attitudes are mostly centered around {\em exemplars} of
good scientific practice as presented in the textbook literature, but
classical texts, specific methodological convictions or even
ontological commitments can also become important for defining a
paradigm. Furthermore, paradigms comprise shared convictions as well
as unspoken assumptions of the group of researchers
\citep[postscript]{kuhn:1976}. An important function of paradigms is
that they both define and limit what counts as relevant question and
legitimate problem within a scientific discipline.

Kuhn's concept of a paradigm is closely connected with his view of how
science develops. According to Kuhn phases of {\em normal science}
where science progresses within the confinements of a ruling paradigm
are followed by {\em scientific revolutions} which, in a process of
creative destruction, lead to a paradigm-shift. Scientific revolutions
are triggered by the accumulation of problems that are unsolvable
within the ruling paradigm (so called {\em anomalies}).  With an
increasing number of anomalies scientists grow unsatisfied with the
current paradigm and start to look for alternatives -- a state of
affairs that \citet[ch. 7/8]{kuhn:1976} describes as the {\em crisis}
of the ruling paradigm. Then, a paradigm-shift can occur that consists
in a thoroughgoing conceptual reorganization of a scientific
discipline or, as the case may be, the genesis of a new
sub-discipline. Unless there is a crisis, the search for alternative
paradigms is usually suppressed by the scientific community.

This theory could be relevant for computer simulations and their
validation. Because computer simulations are sometimes characterized
as a revolutionary new tool that blurs the distinction between model
and experiment, the question can be asked if this tool brings about or
requires new paradigms of validation. Under {\em validation} I
understand a process which allows to test whether the results of a
scientific procedure adequately capture that part of reality which
they are meant to explain or to enable us to understand. It is widely
accepted that for theories or theoretical models, the process of
validation consists in the empirical testing of their consequences by
experiment or observation, which in this context is also often
described as {\em verification} or {\em falsification} or, more
generally, as {\em confirmation}.\footnote{In the realm of computer
simulations the term {\em verification} is, somewhat confusingly,
reserved for checking wether the simulation software is free from
programming errors (so called ``bugs'') and whether it is faithful to
the mathematical model or theory on which it is based. The term {\em
validation} is used for the empirical testing of the simulation's
results. See also Chapter 4 \citep{murray-smith:2019} in this volume.}
The question then is, if the same still holds for computer
simulations, that is, if computer simulations also require some form
of empirical validation before they can be assumed to inform us about
reality.

For the purpose of this paper, I understand empirical validation in a
somewhat wider sense that does not require strict falsification, but
merely any form of matching theoretical assumptions with empirical
findings. In this sense, a historian checking an interpretation
against the historical sources can also be said to validate that
interpretation. However, I assume that proper validation always
includes an empirical component and I therefore use the terms
``validation'' and ``empirical validation'' interchangeably in the
following.

In the following, I first summarize Kuhn's philosophy of science
(Sec.~\ref{sec:kuhn}). Then I list some of the dramatic changes that
computer simulations have brought about in science and -- in order to
forestall possible misunderstandings -- explain why these changes are
not scientific revolutions in the sense of Kuhn (Sec.~\ref{sec:rev}).
In the main part of this chapter (Sec.~\ref{sec:val}), I then examine
the validation of simulations from a Kuhnian perspective. Relating to
the discussion about the relation between computer simulations and
experiments I argue that computer simulations can clearly be
distinguished from real experiments and, therefore, do not require a
new paradigm of validation. In principle, validating simulations is
just like validating theory. I continue by examining whether computer
simulations aggravate the problem of theory choice that is associated
with the so called ``Duhem-Quine-thesis'' \citep{harding:1976}, which
I deny. Finally, I examine some of the issues that the validation of
social simulations and in particular agent-based-models raises from
the point of view of Kuhn's philosophy of science. For the lack of
commonly accepted standards of validation, it seems unclear whether
this field has already reached a state of ``normal science'' with
established paradigms of validation. Because the practices of
validation vary greatly in this field, a general conclusion is not
possible, however. I therefore confine myself to discussing the issue
with respect to selected examples.

\section{Kuhn's philosophy of science}

\label{sec:kuhn}

A crucial aspect of Kuhn's concept of scientific revolutions is the
alleged {\em incommensurability} of paradigms \citep[ch. 12,
postscript 5.]{kuhn:1976} \citep[ch. 2]{sismondo:2007} \citep[sec.
4.3f.]{bird:2013}. Incommensurability means that theories rooted in
different paradigms cannot easily be compared with respect to their
scientific merits, because of

\begin{enumerate}

  \item {\em methodological incommensurability}, which means that the
    criteria of evaluation depend on and change with the paradigm,
  \item the {\em theory-ladenness of observation}, due to which an
    assessment based on empirical evidence may not be able to resolve
    the dispute,
  \item {\em semantic incommensurability}, which means that the
    differences of the respective conceptual reference frameworks and
    taxonomies may render the translation between the nomenclatures of
    different paradigms difficult and error-prone.

\end{enumerate}

Kuhn did not go as far as the proponents of the strong programme of
sociology of science who maintain that the resolution of
interparadigm-disputes is primarily, if not exclusively, determined by
social factors such as group allegiance and power-structures
\citep[sec. 6.3]{bird:2013}. However, he did deny that the choice
between different theories is guided by a scientific meta-method such
as systematic falsification or by any other particular set of rules.
In this respect one can describe Kuhn's stance as a {\em mild
relativism}.  Kuhn's relativism is restricted by his belief that a
common ground for theory choice can still be found in such general
characteristics as empirical accuracy, consistency, breadth of scope,
simplicity or parsimony, fruitfulness for future research \citep[ch.
13]{kuhn:1977}. And he furthermore holds that the comparison and
mutual evaluation of paradigms is possible on the pragmatic basis of
their problem-solving capacity.

Although Kuhn regarded scientific revolutions and the paradigm shifts
they bring about as scientifically perfectly legitimate processes,
that is processes that are primarily driven by a scientific motivation
and not just by social power, he nonetheless found that in almost any
paradigm change some things get lost -- if only that certain questions
will not be considered worthwhile any more. An example is the question
how physical bodies influence each other over a distance, which cannot
be answered by Newton's theory of Gravity and therefore simply was not
asked any more, although, before Newton it was considered important
\citep[ch. 12]{kuhn:1976}. The phenomenon that accepted questions,
problems and even solutions can become orphaned after a paradigm shift
has subsequently been called {\em Kuhn loss} \citep[sec.
2]{bird:2013}.

Also, even though Kuhn allowed for paradigm-shifts to make sense
scientifically, this does not always need to be the case, but one
should expect that sometimes paradigm-shifts are primarily due to
social factors. Not in the least because of the popularity of Kuhn's
theory of scientific revolutions, it has become seductive for
scientists to stage a paradigm shift to promote their scientific
agenda. In order to distinguish illegitimate paradigm-shifs
terminologically, the derogatory term {\em scientific imperialism} can
be used, which has been coined to describe the take-over of a branch
of science by a single paradigm \citep{dupre:1994} by unfair means.
Following Kuhn's line of thought the problem solving capacity could be
a criterion by which to qualify a paradigm shift as either legitimate
or imperialistic. Because of the incommensurability issues described
before, an objective judgment about this can, of course, be difficult.

A contemporary of Kuhn that is often mentioned in the same breath, is
Paul Feyerabend, who is (in-)famous for the slogan ``anything goes''.
In popular folklore this is sometimes understood as meaning that
Feyerabend advocated that in science any method is as good as any
other. However, what Feyerabend actually demonstrated in his book
``Against Method: Outline of an Anarchist Theory of Knowledge''
\citep{feyerabend:1975} and other works was that even from the most
humble historical beginnings, a serious scientific theory or school of
thought can still emerge. Feyerabend's work gains its thrust from the
fact that he can show that some of the game changers in the history of
science such as, for example, Galileo's theory of motion, violated
accepted scientific standards of their time \citep[ch.
9]{feyerabend:1975}. Just as Kuhn he denies that the historical
development of science is or can be guided by methodological or
epistemological rules. Similar to Kuhn, Feyerabend's philosophy has a
certain relativistic flair, which Feyerabend other than Kuhn was ready
to accept \citep[sec. 5]{preston:2016}.

Nonetheless, despite of what the subtitle of his major work suggests,
Feyerabend's analyses do not warrant a strong relativism. Almost all
of Feyerabend's examples concern theories that -- later in their
historical development -- would be considered as scientific even by
conventional standards. Thus, what we can learn from Feyerabend is a
certain tolerance against the methodological chaos of new scientific
approaches in their infant stages. This can be important, for example,
when evaluating social simulations, which according to some authors
suffer from a lack of proper empirical validation
\citep{heath-et-al:2009}. The question is then not so much whether
these simulations adhere to a particular scientific standard but
rather whether the respective scientific community learns from its
failure to do so and will be able to develop appropriate
methodological standards in the future.

Another point that deserves clarification, because it is -- at least
in the philosophical discussion -- almost habitually mentioned in
context with Kuhn, is the {\em Duhem-Quine-thesis}
\citep{harding:1976}. The Duhem-Quine-thesis draws on the fact that if
the logical consequence of a whole system of premises turns out to be
false then it is still unclear which one or more of the premises are
false.\footnote{See also chapter 39 \citep{lenhard:2019} in this
volume.} This means that if a theory is empirically disconfirmed, we
do not (yet) know which part of the theory is wrong. The
Duhem-Quine-thesis can be seen as supporting a certain degree of
arbitrariness, if not relativism in theory choice. And it corresponds
well to Kuhn's view that the way scientists cope with anomalies is not
strictly guided by methodological rules. It may be a matter of
creative choice. As we shall see later, this choice is in practice
much less arbitrary than it may appear in the formal logical
representation of a theory as a system of propositions.

Despite all reservations, Kuhn's picture of the history of science is
still one of linear development, where normal science and
revolutionary phases follow each other in time. For Kuhn the prolonged
co-existence of several competing paradigms was the mark of a
pre-scientific stage where much intellectual energy is wasted in
disputes between rivaling schools of thought. Recent research,
however, has emphasized that the co-existence of different paradigms
within one and the same science is much too common to be dismissed as
pre-scientific \citep{kornmesser:2014, schurz:2014}. This is
particularly true of the social sciences, where hardly ever one
paradigm can claim to solve all puzzles so successfully that it is
able to gather the entire scientific community under its flag. That
Kuhn may have underestimated the amount of co-existence of paradigms
in science does not invalidate his analyses, though. The concepts of
{\em normal science} and {\em scientific revolutions} can still be
employed as ideal-types to characterize the scientific proceedings
within an established paradigm on the one hand and the discourse
between different co-existing paradigms on the other hand.


\section{A revolution, but not a Kuhnian revolution: Computer
simulations in science}

\label{sec:rev}

Kuhn's theory of scientific revolutions is so popular that his concept
of a paradigm has by now become part of the common vocabulary.
Inevitably, it is often used in a sense that is different from what
Kuhn had in mind. It may therefore help to make clear what is not a
revolution or paradigm change in Kuhn's sense. A most salient example
in this context is that of the introduction of computer simulations to
science, because it can with some justification be said that computer
simulations have revolutionized many areas of science. % But were
these really Kuhnian revolutions?

% In a colloquial sense computer simulations have already
% ``revolutionized'' many areas of science in the sense that they
% changed the theory, practice, education in this area. Even the
% borders between different scientific fields become blurred, when,
% for example, construction engineers simulate cancer treatments in
% medicine \citep{simtech1}.  Does this mean that computer simulations
% have prompted a scientific revolution in the sense of Kuhn in the
% respective fields of science?

Computer simulations can roughly be defined as the imitation of a
natural process (or, in the case of social simulations, a social
process) by a computer program \citep{hartmann:1996}. Undoubtedly,
computer simulations have brought about considerable changes in
scientific practice and theoretical outlook. Here are but some
examples:

\begin{itemize}
  \item In engineering, simulations have been used before long to
    simulate the properties of machinery and processes. A large class
    of simulations is based on the method of finite elements which has
    as far reaching applications as structural engineering, car crash
    tests and even cardiovascular simulations
    \citep{carusi-et-al:2013}.

  \item In chemistry simulations are employed in order to simulate
    chemical processes on a quantum-mechanical bases, some of which
    are even outside the reach of direct experimentation
    \citep{arnold-kaestner:2013}.

  \item In climate science the simulations are used to simulate the
    possible future development of the world climate. Naturally,
    experimentation with the world climate is not possible. By the
    same token, unfortunately, these simulations cannot be validated
    directly.

  \item The theory of non-linear dynamical systems (``chaos theory'')
    can even be said to owe much of its origin to computational
    methods \citep{gleick:2011}. At any rate its development has
    certainly been propelled by the use of computers, though it might
    not necessarily have been computer simulations in the narrower
    sense of imitations of a natural process in the computer.

  \item In social science there exists a now already long standing
    tradition of simulating social processes. However, the social
    simulations community still struggles for the acceptance within
    the broader social sciences community
    \citep{squazzoni-casnici:2013}.
\end{itemize}

Some of these examples certainly warrant the characterization as
``revolutionary''. Are they revolutionary in a Kuhnian sense, though?
And would it be reasonable to call simulation-based science in general
a new paradigm of science?

% As far as this is concerned it must be pointed out that Kuhn used
% the terms {\em scientific revolution} and {\em paradigm} in a
% strictly defined sense.
For one thing, the way Kuhn used the term paradigm, paradigms are
always tied to specific scientific disciplines. Even though we are not
tied to Kuhn's definition and the term {\em paradigm} has indeed been
used more liberally by other authors since its original introduction,
it would appear a bit vague to speak of a paradigm of computer
simulations, because it is not at all clear what would be the content
of this paradigm.

Even more importantly, Kuhn reserves the concept of scientific
revolutions for changes that are caused by a crisis of the conceptual
framework of a scientific discipline and that lead to a reconstruction
of the conceptual system that is incommensurable with the previous
reference framework. Not any dramatic change in science is a
revolution in the Kuhnian sense. A prominent example for a dramatic
change that is not a Kuhnian revolution is the discovery of the
structure of the DNA-molecule by Watson and Crick. While this
discovery was a door-opener for molecular genetics, it neither
required nor effected a conceptual reconstruction and there was no
question of it being incommensurable with the previously held views on
hereditary biology. Quite the contrary, it fit in nicely with the
existing body of knowledge. The discovery of the DNA was normal
science at its best, not a Kuhnian revolution.

Similarly, the introduction of computer simulations into a particular
branch of science alone is not a Kuhnian revolution, no matter how
dramatic the changes in scientific practice and the extension of our
knowledge through computer simulations might be. Only, if the use of
computer simulations leads to a revision of established fundamental
concepts, it is a Kuhnian revolution. A possible candidate from the
list above might be chaos theory, in so far as it has modified the
received picture of causality.


\section{Validation of Simulations from a Kuhnian perspective}

\label{sec:val}

Can Kuhn's concept of paradigm illuminate the validation of computer
simulations? And, if so, how? In the following, I am going to state
several questions that can be raised in this context and then try to
give answers to these questions based on the current discussion on
computer simulations in the philosophy of science. The questions that
in my opinion deserve consideration are:

\begin{enumerate}

  \item Notwithstanding the question (discussed earlier) to what
    extent computer simulations have prompted paradigm shifts {\em in}
    science, another question is, whether computer simulations have
    lead to, or require new paradigms in the logic of scientific
    discovery. Classical research logic assumes a clear distinction
    between theoretical research based on deductive inference and
    empirical research based on experiment and (potentially
    theory-laden) observation.\footnote{Because theory-ladenness of
    observation is an often misunderstood topic, two remarks are in
    order: 1) Theory-ladeness of observation as such does not blur the
    distinction between theory and observation. At worst we have a
    distinction between pure theory (without any observational
    component) and theory-laden observation. 2) Theory-ladeness of
    observation does not lead to a vicious circle when confirming
    theories by empirical observation. This is true, as long as the
    observations are not laden with the particular theories for the
    confirmation of which they are used. -- There are areas in science
    where no sharp distinction between theoretical reasoning and
    reporting of observations is made. However, as far as computer
    simulations are concerned, it is clear that because Turing
    Machines do not make observations, a computer program is always a
    theoretical entity - notwithstanding the fact that a computer
    program may represent an empirical setting or make use of
    empirical data. In the latter respect it can be compared with a
    physical theory that may in fact represent empirical reality as
    well as contain natural constants (i.e. empirical data).} Most
    importantly, there is a hierarchy between the theoretical and
    empirical realm. Theoretical assumptions are confirmed or
    disconfirmed by empirical tests -- not the other way round.
    Computer simulations are sometimes depicted as being located
    somewhere between empirical and theoretical research, and -- as
    the common metaphor of ``computer experiments''
    suggests\footnote{See also chapter 37 \citep{beisbart:2019} in
    this volume.} -- blurring the lines between the two
    \citep{morrison:2009}.

  \item In a similar vein, computer simulations often rely on a rich
    mixture of assumptions and technicalities that are drawn from
    diverse sources. In the philosophical literature on simulations
    this has been described as their being
    ``motley''\citep{winsberg:2015} and not simply falling from
    theory. This can raise worries concerning the prospects of
    empirical validation of computer simulations. In particular, the
    question can be asked if the sort of problems associated with the
    Duhem-Quine-thesis increase with computer simulations: You may
    know that your simulation contains many abstractions,
    simplifications and presumptions, but you cannot be sure which of
    these are potentially dangerous.

  \item Finally, some thoughts shall be given to the validation of
    simulations in the social sciences. Because the social sciences
    are multi-paradigm-sciences the validation of simulations raises
    specific problems in this area. Given that it is still not common
    practice to validate simulations, one can even ask whether the
    field of social simulations has already emerged from a
    pre-scientific state.

\end{enumerate}


\subsection{Do computer simulations require a new paradigm of
validation?}

While Kuhn's theory of scientific revolutions is mainly concerned with
the supersession of scientific theories, his concept of paradigms can
also be applied to other aspects of scientific practice. For example,
it might be applied to changes in the logic of scientific research.
The question whether computer simulations bring about (or require) a
new kind of research logic is particularly salient, because it has
been argued recently that computer simulations somehow blur the line
between models and experiments \citep{winsberg:2009}. But if this
means that computer simulations are -- just like experiments --
somehow empirical, the question naturally arises whether the
validation of computer simulations can still be understood along the
lines of what has earlier been described as classical research logic.
Or, if a new paradigm of validation is necessary to assess whether a
simulation adequately captures its target system or not?

Before the recent discussion about the relation of simulations and
experiments, this question seemed to be rather trivial and its answer
obvious: Computers are calculating machines and computer simulations
are nothing but programmed mathematical models that run on the
computer. Therefore, computer simulations can just like models produce
no other than purely inferential knowledge, that is, knowledge that
follows deductively from the premises built into the simulation. In
particular, computer simulations cannot produce genuine empirical
knowledge like experiments or observations can. It is true that
computer simulations can produce new knowledge, because they yield
logical consequences of the built-in premises that were not formerly
known to us \citep[sec. 1.3.4]{imbert:2017}. It is also true that
computer simulations can -- like any model -- produce knowledge about
empirical reality, because the premises built into them have empirical
content and so have their logical consequences. But this is far cry
from the empirical knowledge that experiments or observations yield
and which -- because it is of empirical origin -- is genuine. But then
computer simulations have just the same epistemic status as theories
and models and therefore follow the same research logic and require
just the same kind of validation. Now, in order to validate a model or
a theory it must be tested empirically, and so must computer
simulations.

What I have just described is more or less the picture of computer
simulations that was pertaining in the general literature on
simulations up to the beginning of the millennium. It had by that time
been fleshed out with two distinctions that make the difference
between computer simulations and empirical research procedures
extraordinarily clear: Firstly, by the distinction of the modus
operandi. Is it a {\em formal} procedure (computer simulation) or a
{\em material} process (experiment)? Secondly, by the distinction of
their relation to the target system. Accordingly, this relation could
be characterized as one of {\em formal similarity} \citep{guala:2002}
with the object of the simulation being a {\em representation}
\citep{morgan:2003} of the target system or, in the case of
experiments, one of {\em material similarity} with the object of
experimentation being a {\em representative} of the target system.

In recent years, however, there has been a persistent discussion among
philosophers of science during the course of which the distinction
between simulations and experiments has been seriously called into
question. Most notably, some authors have claimed that it is
impossible to make a sharp distinction between simulations and
experiments -- at least as far their epistemic reach or inferential
power is concerned. \citep{winsberg:2009, parker:2009, morrison:2009,
winsberg:2015}. Others have advocated the weaker claim that while
there is a distinction between the two categories, the transition
between them is smooth and that there are borderline cases for which
it is difficult to determine into which category they fall
\citep{morgan:2003}.

Now, if this were true, then the generally accepted research logic of
empirical science, which relies on the ability to distinguish clearly
between empirical observation and theoretical reasoning would find
itself in a serious crisis and we would have to expect and, in fact,
need to hope for new paradigms of research logic and, in particular,
for the validation of computer simulations to emerge.

\begin{figure}
\doublespacing
\begin{center}
\begin{scriptsize}
\begin{tabular}{l|c|c|c|}
  \multicolumn{1}{c}{ } & \multicolumn{1}{c}{ } & \multicolumn{2}{c}{$\overbrace{\hspace{7cm}}^{Experiments}$} \\ \cline{2-4}
                      & \textbf{computer simulation} & \textbf{analog
                      simulation} & \textbf{real experiment} \\ \hline
materiality of object & semantic              &
                      \multicolumn{2}{c|}{material} \\ \hline
relation to target & \multicolumn{2}{c|}{representation (formal
			          similarity)}       & representative \\ \hline
\multicolumn{1}{c}{ } &
\multicolumn{2}{c}{$\underbrace{\hspace{8cm}}_{Simulations}$} &
\multicolumn{1}{c}{ } \\
\end{tabular}
\end{scriptsize}
\end{center}
\caption{\small Conceptual relation of simulations and experiments \citep{Arnold2013c}}\label{SimulationExperimentsSchema}
\end{figure}


However, the case for the non-discriminability of simulations and
experiments rests almost entirely on conceptual confusions and an
ambiguous use of the term ``experiment''. The examples with which
supporters of the non-disicriminabilty thesis demonstrate their claim
concern almost exclusively atypical kinds of experiments, where the
object of experimentation is not really a representative of the target
system. For example, \citet[590]{winsberg:2009}, discusses ``tanks of
fluid to learn about astrophysical gas-jets'' as an instance of an
experiment. But this is an atypical experiment, because the tanks of
fluid are not representatives of the target system (astrophysical
gas-jets). This kind of experiment is indeed in no better position to
produce genuine empirical knowledge about the target system than any
computer model. But the fact that there are such atypical experiments
does not contradict the fact that there exist real experiments that
can produce genuine empirical knowledge about their target system and
that this is a feature that distinguishes real experiments from
models.

The conceptual confusion that exists in the philosophical discussion
about the relation of simulations and experiments can easily be
clarified by the schema on figure \ref{SimulationExperimentsSchema},
which depicts the overlap in the use of the words ``simulation'' and
``experiment''. The kind of experiments that Winsberg and other
authors advocating the non-discriminability between simulations and
experiments discuss over and over again, has been termed ``analog
simulation'' in the schema. As all experiments do, ``analog
simulations'' operate on a material object, but this object does not
have a material similarity to its target system and therefore is only
a representation, but not a representative of its target system.  The
latter is required for an experiment to produce genuine empirical
knowledge about its target system.

That simulations are not experiments -- save for the ambiguity and
overlap in the use of words -- becomes furthermore clear if we
consider the kind of experiments that give rise to anomalies and which
in retrospect are declared crucial experiments that decide the choice
between conflicting theories. Because the laws of the scientific
theories are programmed into computer simulations, they cannot be used
to test these very theories. If it really was as difficult to
distinguish between simulations and experiments as some philosophers
of science believe, then it should -- at least in principle -- be
possible to substitute experiments with simulations in any context.

However, if we draw the demarcation-line between analog simulations
and real experiments and not, as the authors advocating the
non-discriminability-thesis implicitly do, between computer
simulations and analog simulations, then we are able to distinguish
clearly those scientific procedures that can generate genuine
empirical knowledge about their target system from those that cannot.
Simulations and, in particular, computer simulations belong to the
latter category and therefore have -- with respect to validation --
the same epistemic status as theories and models. They need to be
validated empirically, but they cannot provide empirical
validation.\footnote{In simulation-science the term {\em empirical} is
sometimes used to distinguish simulation and numerical methods from
mathematical analysis. (\citet{phelps:2016} is an example of this.)
But this is just a different use of words and should not be confused
with ``empirical'' in the sense of being observation-based as the word
is understood in the context of empirical science.}

Summing it up, computer simulations do not break the received paradigm
of research logic of empirical science. Therefore, a new paradigm of
validation specifically for simulations is not needed.

\subsection{Validation of simulations and the Duhem-Quine-thesis}

Another point frequently emphasized in the philosophy of simulation
literature is that computer simulations can become highly complex.
This is also one of the major differences between computer simulations
and thought experiments, to which they are otherwise quite similar. At
least in the natural sciences computer simulations can often be based
on comprehensive and well tested theories, such as quantum mechanics,
general relativity, Newton's of gravitation or -- in engineering --
the method of finite elements. But even in the natural sciences
simulations cannot always be based on a single theory, but they
sometimes rely on different theories from different origins. Climate
simulations are a well-known example for this. And even where
simulations are based on a single theory, they usually also draw on
various sorts of approximations, local models and computational
techniques. None of these can be derived from theory, so that they
need independent credentials. This situation has been described in the
philosophy of simulation literature as their being motley and partly
autonomous \citep{winsberg:2003}. This description echos a recent
trend in the philosophy of science which emphasizes the importance and
relative independence of models from theory
\citep{morgan-morrison:1999, cartwright:1983}.

So, if simulations are knit together from many independent set pieces
of theories, models, approximations, algorithmic optimizations etc.,
then the Duhem-Quine-thesis could point out a potential problem. A
possible reading of the thesis assumes that if validation fails (for
example, because an empirical prediction was made that turned out to
be wrong), then one cannot know which part of the chain of theoretical
reasoning failed that leads to the empirical prediction. In the case
of computer simulations this means that one does not know whether the
theory on which the simulation is based, the simplifications that may
have been made in the course of modeling or, finally, the program code
has failed.

By the same token, if this reading of Duhem-Quine is accurate,
simulation scientists would -- for better or worse -- enjoy a great
freedom of choice concerning where to make adjustments if a simulation
fails, i.e. if it leads to unexpected, obviously false or no results
at all. Some philosophers have even argued that scientists sometimes
deliberately employ assumptions that are known to be false to make
their simulations work. Among these are artificial viscosity
\citep[sec. 8]{winsberg:2015}, or -- another often cited example --
``Arakawa's trick'' \citep{lenhard:2007}. Arakawa based a general
circulation model of the world climate on physically false assumptions
to make it work, which by the scientific community was accepted as a
technical trick of trade.

However, this reading of Duhem-Quine paints a somewhat unrealistic
picture of scientific practice, because in case of failure there
usually exist further contextual cues where the error causing the
failure has most likely occurred. While in the abstract formal
representation of theories that is sometimes used to explain
Duhem-Quine, the premises are represented as propositions with no
further information, scientists usually have good reasons to consider
the failure of some premises as more likely than others. In science
and engineering, the premises are usually ordered in a hierarchy that
starts with the fundamental physical, chemical or biological theories,
ranges over various steps of system description and approximation down
to the computer algorithms and, ultimately, the programm code. If a
simulation fails one would start to examine the premises in backward
order. And this is only reasonable, because prima facie, it is more
likely that your own program code contains a bug than, say, that the
theory of quantum mechanics is false or that some of the tried and
tested approximation-techniques are wrong. Though, of course, this is
not completely out of the question, too.\footnote{See \citet[sec.
3.4]{arnold-kaestner:2013} for a case-study containing a detailed
description of this hierarchy of premises.} It should be understood
that the credibility of the various premises occurring in this
hierarchy does not follow their generality, but depends on their
respective track record of successful applications in the past. It can
safely be assumed that this situation is typical for normal
science.\footnote{But see \citet{lenhard:2019} in chapter 39 in this
book, who paints a very different picture. I cannot resolve the
differences here. In part they are due to Lenhard using examples where
`` 'due to  interactivity, modularity  does not  break down  a
complex  system  into separately manageable  pieces.' '' To me it
seems that as far as software design goes, it is always possible --
and in fact good practice -- to design the system in such a way that
each unit can be tested separately. As far as validation goes, I admit
that this may not work as easily because of restrictions concerning
the availability of empirical data.}

It must be conceded, though, that during a scientific revolution or
within cross-paradigm-discourse, there might be no hierarchy of
premises to rely on, because some of the premises higher up in the
hierarchy, like the fundamental theories, are not generally accepted
any more. In this situation, there might, as Kuhn suggested, only be
vague meta-principles left to rely on and we must face the possibility
of not being able to resolve all conflicts of scientific opinion.

What about the conscious falsifications like artificial viscosity and
``Arakawa's trick'' that -- according to some philosophers of science
-- are introduced by simulations scientists in order to make their
simulations work? This reading has not gone unchallenged, and it has
been called in to question whether the artificial viscosity that
Winsberg mentions is more than just another harmless approximation
\citep{peschard:2011b} or whether ``Arakawa's trick'' not merely
compensates for errors made at another place, which would make it an
example of a simulation the success of which is badly understood
rather than one that is very representative of simulation-based
science \citep[333f.]{beisbart:2011}. It seems that these
philosophically certainly interesting examples concern exceptions
rather than what is the rule in the scientific practice with
simulations. For the time being that is to say, because it is well
imaginable that in the future development of science these tricks
become more common.

Summing it up, with respect to the Duhem-Quine-thesis there are
neither additional challenges nor additional chances for the
validation of simulations. Under {\em normal science}-conditions it
does not play a role at all. Other than that it merely reflects the
greater methodological imponderabilities during a revolutionary phase
or in an inter-paradigm context.

\subsection{Validation of social simulations}

Most of the discussion so far and all of the examples were centered
around science and engineering. Therefore, in the following I am going
to briefly discuss questions concerning the validation of simulations
that are more specific for the social sciences.

\subsubsection{Where social simulations differ}

In the context of validation of social simulations two features of the
social sciences become relevant that distinguish them from most
natural sciences: Firstly, the social sciences are
multi-paradigm-sciences. It is the normal state of these sciences that
there exist multiple more or less mutually incommensurable paradigms
at the same time. This multi-paradigm-character is well described in
the textbook by \citet{moses-knutsen:2012}. For Kuhn such a state of
affairs was a sign of a pre-scientific phase. But given that the
social sciences are -- within inevitable confinements -- nonetheless
able to produce convincing explanations at least for some social
phenomena, the qualification as pre-scientific seems inadequate. Also,
if considered in isolation, most of these paradigms expose typical
features of normal science, like a textbook-literature, role models
and exemplars etc.

Deviating from Kuhn, I therefore suggest, that the qualification as
pre-scientific should be reserved to those sciences or branches of a
science that -- given their state of development -- have not yet been
able at all to produce results that can be validated or confirmed by
some reasonable procedure. The qualification as pre-scientific is in
so far justified as without a common understanding and practice of
validation one can never be sure whether the results are indeed
reliable.

Secondly, the social sciences include qualitative paradigms, including
paradigms that rely on hermeneutical methods. It is safe to assume
that these can neither be completely ignored nor always be resolved to
quantitative or otherwise formal methods and paradigms.\footnote{There
are scientists who deny even this and who also believe that without
formal models no explanation of any sort is possible in history or
social science. I am a bit at loss for giving proper references for
this point of view, because I have mostly been confronted with it
either in discussions with scientists or by anonymous referees of
journals of analytic philosophy. The published source I know of that
comes closest to this stance is the keynote ``Why model?'' by Joshua
\citet{epstein:2008}, which I have discussed in \citet{arnold:2014}.}
As computer simulations are quantitative, the decision to use computer
simulations is also a decision for a quantitative paradigm.

Here, I understand the term ``quantitative'' in a wide sense,
including anything that is described in a formal language. This can be
formal logic, mathematics, or a programming language. This wide sense
of using the term ``quantitative'' is motivated by the fact all formal
descriptions share the same epistemic risks of either losing important
information, because the expressive power of formal languages is
limited in comparison to natural language, or adding arbitrary
assumptions in form of modeling decisions. A simulation model forces
its author to provide detailed mechanics of all processes that are
included in the model, because otherwise the model would not run.
However, if the mechanics are not known, this amounts to theoretical
speculation. A purely verbal description, in contrast, allows its
author to remain silent or at least adequately vague about underlying
mechanics the details of which are not known. On the other hand,
because of their strict specification, formal models cannot as easily
be misunderstood as verbal descriptions. And they enforce logical
consistency.

Both of these features affect the validation of social simulations.
Because, when trying to validate a simulation study, say, on the
evolution of cooperation, it might become necessary to compare its
findings with those of biological field research or, depending on the
envisaged application cases, those of cultural history. Thus,
different scientific disciplines with different paradigms might be
affected. And, it might become necessary to translate between a
qualitative descriptive language used in empirical research and the
formal languages used in simulation research.

One possible objection when discussing social simulations in the
connection with Kuhn, is that it is not a scientific discipline, but a
field that runs across several disciplines. However, since this field
is shaped by shared attitudes, well-known exemplars
\citep{axelrod:1984, axtell-et-al:2002, epstein-axtell:1996,
schelling:1971} and an emerging textbook-literature
\citep{railsback-grimm:2012, gilbert-troitzsch:2005}, looking at it
from a Kuhnian perspective does not seem too far-fetched.

\subsubsection{Are social simulations still in a pre-scientific stage?}

One of the most surprising features to the outside observer of the field of
social simulations in general is the widespread absence of empirical
validation, sometimes combined with a certain unwillingness to see this as a
problem.

In a meta-study on agent-based-modeling (ABM), which is one very
important sub-discipline of social simulations,
\citet{heath-et-al:2009} find that the models in 65\% of surveyed
articles have not properly been validated, which they consider ``a
practice that is not acceptable in other sciences and should no longer
be acceptable in ABM practice and in publications associated with
ABM'' (4.11). While some of these not-validated simulations can serve
a purpose as thought experiments that capture some relevant connection
in an idealized and simplified form \citep{reutlinger-et-al:2017},
many of them are merely follow-ups to existing simulations and bear
little relevance of their own. The practice of publishing simulations
without empirical validation and seemingly little (additional)
theoretical relevance is so widespread that it has been termed the
YAAWN-Syndrome where YAAWN stands for "Yet Another Agent-Based Model
... Whatever ... Nevermind" \citep{osullivan-et-al:2016}. The fact
that such a term has been coined is an indication that the
ABM-community is growing weary of unvalidated or otherwise
uninteresting simulations. Thus, the situation may change in the
future. For the time being, lack of validation is still a problem.

To be sure, agent-based-modeling is a broad field. On the one hand
side there are very theoretical simulations that set out from abstract
concepts but without any particular application case in mind. And on
the other hand side there exist simulations that are right from the
start related to a particular empirical setting. The latter kind of
simulations is typically found in corporate or political consulting. I
am going to look at the theoretical simulations first and then
consider the more applied kinds of simulations later.

Naturally, unvalidated simulations are much more prevalent among the
theoretical simulations, where the lack of empirical validation is
sometimes not even perceived as a problem. This may be illustrated by
a quotation from an interview with a philosopher who has produced
models of opinion dynamics \citep{hegselmann-krause:2002} that have
frequently been cited in other modelling-studies but that have not
been empirically validated:

\begin{quotation}
  None of the models has so far been confirmed in psychological
  experiments. Should one really be completely indifferent about that?
  Rainer Hegselmann becomes almost a bit embarrassed by the question.
  ``You know: In the back of my head is the idea that a certain sort
  of laboratory experiments does not help us along at all.'' \citep[p.
  2]{groetker:2005}
\end{quotation}

But if laboratory experiments do not help us along, how can models
that have never been confirmed empirically either by laboratory
experiments or by field research help us along? This lack of interest
in empirical research is all the more surprising as opinion dynamics
concern a field with an abundance of empirical research. Naively, one
should assume that scientists have a natural interest in finding out
whether the hypotheses, models and theories they produce reflect
empirical reality. That this is obviously not always the case,
confirms Kuhn's view that the criteria by which scientific research is
judged are also set by the paradigm that guides the thinking of the
researchers and that there is no such thing as a ``natural''
scientific method independent of paradigms. However, even Kuhn's mild
relativism would rule out science without any form of empirical
validation as unrewarding.

The lack of empirical concern within the field of social simulations
can furthermore be attributed to another working mechanism of
paradigms that Kuhn identified, namely, the role of {\em exemplars}.
As mentioned earlier, according to Kuhn scientific practice is not
guided by the abstract rules of a logic of scientific discovery.
Instead, scientists follow role models or {\em exemplars} of good
scientific practice.

Some very influential role models in the field of social simulations
concern simulations that have never successfully been validated. The
just mentioned opinion-dynamics simulation by Hegselmann and Krause is
one example for this kind of role model. But the arguably most famous
unvalidated model that serves as an exemplar in Kuhn's sense is Robert
Axelrod's ``Evolution of Cooperation'' \citep{axelrod:1984}. Despite
the fact that the reiterated Prisoner's Dilemma simulations that
Axelrod used as a model for the evolution of cooperation had turned
out to be a complete empirical failure by the mid 1990s
\citep{dugatkin:1997} and despite the devastating criticism Axelrod's
approach had received from theoretical game theory
\citep{binmore:1994, binmore:1998}, it continues to be passed down as
a role model of social simulations until this day. In a journal
article from 2010 in the prestigious {\em Science}-journal, where a
similar research design as Axelrod's was employed, it is mentioned as
a role model that has been ``widely credited with invigorating the
field'' \citep[2008f.]{rendell-et-al:2010a}. And one can easily find
recent studies \citep{phelps:2016} that naively pick up Axelrod's
study as if no discussions concerning its robustness, its empirical
validity or its theoretical scope had ever taken place in the
meantime. If simulation-research-designs without proper validation
such as Axelrod's continue to be treated as exemplars, it is no
surprise that many social simulations lack proper validation.

Now, there are two caveats: Firstly, in some cases unvalidated
simulations can serve a useful scientific function, among other things
as thought-experiments. Of a thought experiment one usually does not
require empirical validation. Thus, if Axelrod's evolution of
cooperation or Hegselmann's and Krause's opinion dynamics could be
considered thought experiments their status as role models in
connection with their lack of empirical validation could not be taken
as an indication that social simulations still remain in a
pre-scientific stage. However, the way that both these simulations
functioned as role models was not by their (potential) use as thought
experiments, but as a research programme. Indeed, it would be hard to
justify the literally dozens if not hundreds of follow-up simulations
to Hegselmann-Krause or Axelrod as thought experiments without
invalidating the category of a thought experiment as a useful
scientific procedure. But it has to be kept in mind that not any kind
of unvalidated simulation is an indication of pre-scientific fiddling
about.

Secondly, and more importantly, not all simulation traditions have, of
course, remained as disconnected from empirical research as Axelrod's
Evolution of Cooperation and Hegselmann's and Krause's opinion
dynamics simulations. One example is the Garbage-Can-Model (GCM) by
\citet{cohen-et-al:1972} which describes decision making inside
organizations with a four component model, taking ``problems'',
``solutions'', ``participants'' and ``opportunities'' into account.
This model is highly stylized and, because of this, would be difficult
to validate directly. Nevertheless, it is frequently referred to in
studies on organizational decision making, including empirical
studies.

But why, one may ask, could the connection to empirical research, or
more generally, other kinds of research on organizational decision
making be established in this case while it failed in the
aforementioned cases? There are several possible reasons:

\begin{itemize}

  \item Modeling organizational decision making is a much more
  restricted topic than, say, modeling evolution of cooperation in
  general. This makes it easier to find the right abstraction level
  for modeling. While biologists complained about simulations of the
  evolution of cooperation that ``Most repeated animal interactions do
  not even correspond to repeated games'' \citep[p.\
  83]{hammerstein:2003}, researchers from organizational science have
  no such difficulties in relating to the Garbage Can Model in their
  case studies \citep{fardal-sornes:2008, delgoshaei:2013}.

  \item Within organization theory working with stylized descriptions is
  generally accepted. Thus, the target that the simulation-model had to match
  was an already highly stylized verbal description. (Nonetheless, the
  simulation model did not represent the verbal description faithfully
  \citep[1.4]{fioretti:2008}.) It is much easier to cast a stylized verbal
  description convincingly into a simulation-model than, say, a thick
  historical narrative as in one of Axelrod's suggested application cases
  \citep{northcott-alexandrova:2015, arnold:2008}.\footnote{I am indebted to
  Julian Newman for pointing out to me the excellent paper by
  \citet{northcott-alexandrova:2015} on the Prisoner's Dilemma. It contains the
  so far best analysis why Alexrod's reinterpretation in terms of the
  Prisoner's Dilemma of truces in WWI ultimately fails. And because the
  author's have obviously not been aware of my own research on the topic, I
  consider it as an independent confirmation of my own critical conclusions
  regarding Axelrod's chapter on WWI \citep[ch. 5.2.2]{arnold:2008}.}

  \item For the study of organizational decision making the Garbage
  Can Model seems to serve as a kind of vantage point. It helps to
  analyze and communicate organizational decision making problems by
  relating a particular decision making situation to the model -- even
  if the model is only used as a conceptual reference framework and
  the actual simulation results are ignored.\footnote{This seems to be
  the standard case for applying the GCM in organizational science.
  See \citet{fardal-sornes:2008} and \citet{delgoshaei:2013} for
  example. It will be interesting to see whether the more refined
  simulation models of the GCM that have been published more recently
  \citep{fioretti:2008} will bring about an increased use of
  simulation models in applied studies referring to the GCM or not.}
  Because of its popularity the Garbage-Can-Model could even be
  considered an exemplar in Kuhn's sense. To serve as a vantage point,
  a model does not need to be empirically validated or even testable.
  It stands to reason, though, that it still needs to be ``realistic
  enough'' in some weaker sense to serve this purpose.

  % With respect to scientific communication one advantage of formal
  % (i.e. mathematical or programmed) models is that they leave less
  % room for interpretation and misunderstanding than verbal
  % descriptions. This advantage must be weighed against the
  % disadvantage of having less expressive power than natural
  % language, especially when it comes to articulating human behavior
  % and interaction.

  \item While for the latter purpose (vantage point) a stylized verbal
  description could suffice, simulation models have the advantage that
  they can be run. This allows to generate hypotheses about the
  simulated process which
  %\footnote{Arguably, so far the most common use of the CGM has been
  %that of a conceptual reference framework. I am not aware in how far
  %any of the hypotheses that can be generated from running
  %simulations of the model have become relevant for organizational
  %research.}
  can help to establish the basic plausibility of the model, if the
  simulation itself and its results are plausible in view of the prior
  knowledge about the simulated process.\footnote{This is precisely
  where Axelrod's simulations was lacking, because a) his tournament
  of reiterated Prisoner's Dilemmas is too far removed from the
  phenomenology of either animal or human interaction to be prima
  facie plausible, and b) his results were - unbeknownst to him -
  highly volatile with respect to the simulation setup and thus also
  lack plausibility.} In the case of the GCM the model establishes the
  connection between a certain structure of the decision making
  process and certain characteristics of the outcome, like how
  efficiently problems will be solved. In a verbal description this
  connection can be maintained, but not be demonstrated. A simulation
  can show that such a connection exists, even if only within the
  model.

\end{itemize}

In view of the possible functions of communication and
hypotheses-generation, one can argue that models like the Garbage Can
Model can be useful in the context of empirical research even without
being empirically validated themselves. Still, the question remains
what characteristics a model of this kind must have to be considered
useful or suitable, or how one can tell a good model from a bad model.
There seems to exist an intuitive understanding within the scientific
communities habitually using these models, but it is hard to find any
explicit criteria. This strengthens the impression that a paradigm of
validation is not yet in place, at least not for the more theoretical
simulations.

What about applied simulations, though? Agend-based-models are, among
other things, used to give advice about particular policy measures,
like introducing a new pension plan \citep{harding-et-al:2010} or
determining the best procedures for research funding
\citep{ahrweiler-gilbert:2015}. Obviously, validation is of
considerable importance if simulations are used for political
consulting. So, how do scientists who apply social simulations get
around the restriction that the simulation results often cannot
directly be compared with measurable empirical data? In particular,
how can simulations be validated that are meant to evaluate the
possible consequences of policy measures that might never be
implemented?

In their discussion of the validation of the SKIN-model, which
simulates knowledge dynamics in innovation networks, \citet[section
1.1.2]{ahrweiler-gilbert:2015} do not even assume that there exist
objective observations independent of a concrete research goal or
question.\footnote{They discuss this unter the heading of
``theory-ladenness of observations'', though their examples suggest
that the issue at stake is rather different interpretations of
observations or a focus on different observations depending on the
research questions than different observations due to a different
theoretical background.} At least for the sake of the argument they
even accept the view that the observation of a social process is a
construct of this process or ``what you observe as the real world''
\citep[section 1.2]{ahrweiler-gilbert:2015}, just like the simulation
of the same process is another construct of this process. However,
since the authority over what is observed as the real world lies with
the ``user community'' \citep[section 1.3]{ahrweiler-gilbert:2015},
the output of a simulation can meaningfully be compared with the
observations.

Since the construction of the simulation as described by
\citet[section 2.4]{ahrweiler-gilbert:2015} is a process in which the
user community is deeply involved, it is tempting to raise the
question how unbiased this kind of validation really is. After all, an
administration assigning the task of examining the potential for
enhancement of their administrative procedures to a team of simulation
scientists might be more interested in the vindication of certain
administrative procedures than in their unbiased assessment. However,
the ``user community view'' as described by
\citet{ahrweiler-gilbert:2015} depicts only the outline of the
construction and validation process of applied agent-based-models. A
more detailed analysis of the validation of applied agent-based-models
as provided by \citet{harding-et-al:2010} reveals that there exists a
whole array of validation procedures which, if executed properly,
limits the risk of producing biased or arbitrary results. For the
Australian Population and Policy Simulation Model
\citet{harding-et-al:2010} report, among other measures: i) the
calibration and benchmarking of the simulation with available
cross-sectional and longitudinal data, ii) the comparison of the
simulation model's projection with that of other models, iii) the
modular structure and separate evaluation of each module, iv) the
examination, if both the individual agent's simulated life histories
and the summary statistics yield reasonable results. The impact of
proposed policy measures as revealed by the simulation can by its very
nature not beforehand be compared with empirical data. However, one
can contend that in the context of policy advise a simulation is
sufficiently validated, if it leads to policy decisions that are
better grounded than they would be without running a simulation model.

Where does this leave us? Are social simulations still in a
pre-scientific stage with respect to their validation? On the one hand
there is a widespread lack of proper validation and the impression
that the increasing number of published agent-based models does not
necessarily pay off in terms of further deepening our understanding of
the simulated processes. While other quality issues of agent-based
models, such as their reproducibility and mutual comparability, have
been addressed in recent years,\footnote{A most notable initiative in
this respect has been the introduction of the ODD Protocol for the
standardized description of agent-based-models
\citep{railsback-grimm:2012}.} there is still no common understanding
concerning how agent-based models should be validated. So far, the
textbooks on agent-based simulations have little to say about
validation. With the central issue of validation still being
unresolved, the field of social simulations does yet seem to have
matured into a normal science in the sense of Kuhn. The situation can
positively be a described as a phase of humble beginnings in the sense
of the interpretation of Feyerabend's anarchic epistemology that was
given earlier.

On the other hand, scientists that apply agent-based-models to
particular empirical processes typically invest considerable time and
effort into the validation of their simulations and employ a diverse
set of validation procedures to ensure the credibility of their
simulations. So, we might indeed be witnessing a paradigm of
validation of applied agent-based-models in the making. It is, so far,
only in the making, because the various validation procedures and
criteria used by the practitioners do not yet seem to have been
consolidated to a degree where they become textbook knowledge.


\section{Summary and Conclusions}

Putting it all together, we arrive at fairly conservative conclusions:
Kuhn's theory of scientific revolutions and his concept of a paradigm
does not have any particular consequences for the validation of
simulations. At least it does not have consequences that are any
different from those it has for the validation of theories or
non-simulation models. And neither do computer simulations require us
to reconsider Kuhn's theory or related topics like the
Duhem-Quine-thesis. This result is somewhat unspectacular, but it may
be clarifying. With regard to the discussion about the novelty of
computer simulations it means that, whatever the novelty may be,
neither the introduction of computer simulations nor their validation
is or requires a Kuhnian revolution.

The co-existence of multiple paradigms in the social sciences is a
challenge for Kuhn's theory in its original form. But, again, the
validation of simulations does not raise any specific problems in this
context. Presently, many social simulations suffer from the fact that
for the lack of proper validation they are quite uninformative about
their target system. Although, there are also examples where social
simulations do contribute to the understanding of the target system,
the field as a whole does not yet seem to have become normal science
in the sense of Kuhn. This is most notably due to the fact that -- as
of now -- there exists no commonly shared understanding of the
validation requirements of social simulations.

\printbibliography

% \bibliographystyle{apsr}

% \bibliography{bibliography}

\end{document}
